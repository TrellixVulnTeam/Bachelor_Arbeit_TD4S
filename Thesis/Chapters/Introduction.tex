% !TeX root = ../Thesis.tex

%************************************************
\chapter{Introduction}\label{ch:introduction}
%************************************************
\glsresetall % Resets all acronyms to not used

In this thesis, we modify the popular academic open-source hardware-synthesis tool \gls{VPR}\cite{vtr8} to perform in-placement wirelength estimation using \glspl{NN}.

\section{Motivation}

The recent progress in the field of deep \glspl{NN} has motivated their use in a vast variety of fields, from weather prediction, over efficient data compression, to ubiquitous voice assistants. Deep \glspl{NN} have not only enabled the efficient solving of previously hard problems, but also improve many existing systems by often providing faster or more accurate predictions than other categories of \gls{ML}.

One field that employs computationally intensive heuristics, which might profit from the characteristics of \glspl{NN}, but which has not yet gained much attention for research about the benefits of integrating \glspl{NN}, is automated hardware synthesis. While there are many approaches to leverage deep \glspl{NN}, there are also many unexplored areas with respect to this rather recent technology.

We specifically identified the in-placement wirelength estimation as a candidate to deploy \glspl{NN}, as this runtime-critical subroutine can profit from accurate predictions, while currently only using very simple explicit heuristics.

Even though inference in \glspl{NN} will most certainly be orders of magnitude slower than the constant-runtime approximation used in the \gls{VPR} Placer, we hope that a substantial increase in accuracy will mitigate the increased runtime.

\section{Contribution}

Our contribution in this thesis consists of analysing \gls{VPR}, modifying its placement algorithm to use \gls{NN} inference for wirelength estimation, and evaluating the effectiveness of this approach.

We will implement two different types of \glspl{NN}, namely \glspl{RNN} and \glspl{CNN}, and integrate them into \gls{VPR}. A comprehensive comparison with the unchanged \gls{VPR} Placer will then show if this approach is able to improve the performance of \gls{VPR}.

\section{Outline}

This thesis is structured as follows: First, we give an overview of the problem, its domain, and the technologies we deploy in Chapter \ref{ch:relatedwork}. We then outline our approach and chosen methods in Chapter \ref{ch:design}. Implementation and data specific details are given in Chapter \ref{ch:implementation}. We thoroughly evaluate and analyse our system in Chapter \ref{ch:evaluation}, also discussing our results and sketching possible enhancements. This thesis is concluded in Chapter \ref{ch:Conclusions}. Instructions on the installation and usage of our system and additional resources can be found in the Appendices \ref{ch:InstallAndUse} and \ref{ch:AdditionalData}, respectively.